\documentclass{article}

\usepackage{fullpage}
\usepackage{color}
\usepackage{amsmath}
\usepackage{url}
\usepackage{verbatim}
\usepackage{graphicx}
\usepackage{parskip}
\usepackage{amssymb}
\usepackage{nicefrac}
\usepackage{listings} % For displaying code


\def\rubric#1{\gre{Rubric: \{#1\}}}{}

% Colors
\definecolor{blu}{rgb}{0,0,1}
\def\blu#1{{\color{blu}#1}}
\definecolor{gre}{rgb}{0,.5,0}
\def\gre#1{{\color{gre}#1}}
\definecolor{red}{rgb}{1,0,0}
\def\red#1{{\color{red}#1}}

% Math
\def\norm#1{\|#1\|}
\def\R{\mathbb{R}}
\def\argmax{\mathop{\rm arg\,max}}
\def\argmin{\mathop{\rm arg\,min}}
\newcommand{\mat}[1]{\begin{bmatrix}#1\end{bmatrix}}
\newcommand{\alignStar}[1]{\begin{align*}#1\end{align*}}

% LaTeX
\newcommand{\fig}[2]{\includegraphics[width=#1\textwidth]{a0f/#2}}
\newcommand{\centerfig}[2]{\begin{center}\includegraphics[width=#1\textwidth]{#2}\end{center}}
\def\items#1{\begin{itemize}#1\end{itemize}}
\def\enum#1{\begin{enumerate}#1\end{enumerate}}

\begin{document}

\title{CPSC 340 Assignment 0}
\author{Beverly Biwen Low - 58977463}
\date{}
\maketitle

\vspace{-4em}



\section{Linear Algebra Review}


\subsection{Basic Operations}

\enum{
\item 14
\item 0
\item (6 10 14)
\item square root of 5
\item (0 1 2)
\item $\left[\begin{array}{ccc}
3 & 1 & 1\\
2 & 3 & 1\\
2 & 1 & 3
\end{array}\right]$
\item $\left[\begin{array}{c}
6\\
5\\
7\\
\end{array}\right]$
}
\subsection{Matrix Algebra Rules}

\begin{enumerate}
\item True
\item True
\item False
\item False
\item False
\item True
\item False
\item True.
\item True
\end{enumerate}

\subsection{Special Matrices}
\enum{
\item Symmetric matrix is a square matrix that is equal to its transpose.
\item Identity matrix is a square matrix in which all elements of principal diagonal are ones and all other elements are zeros. Any matrix multiplied by it will get itself.
\item Orthogonal matrix is a matrix where the transpose is equal to the inverse.
}

\section{Probability Review}

\subsection{Rules of probability}

\begin{enumerate}
\item 0.25
\item 4 dollars
\item 0.55
\end{enumerate}

\subsection{Bayes Rule and Conditional Probability}

\begin{enumerate}
\item 0.010094
\item False positives
\item 0.00941153
\item Yes. The probability of a person is a drug user given that the test is true is very small.
\item 
\end{enumerate}


\section{Calculus Review}


\subsection{One-variable derivatives}

\begin{enumerate}
\item $\frac{14}{3}$
\item 0.25
\item 0
\item $\frac{-\exp(-x)}{1+\exp(-x)}$
\end{enumerate}

\subsection{Multi-variable derivatives}
\begin{enumerate}
\item ( $2x_1$ , $\exp(x_2)$ )
\item ( $\exp(x_1 + x_2x_3)$ , $x_3\exp(x_1 + x_2x_3)$ , $x_2\exp(x_1 + x_2x_3)$ )
\item ( $a_1$ , $a_2$ )
\item ( 4$x_1$ - 2$x_2$ , 2$x_2$ - 2$x_1$ )
\item ( $x_1$ , $x_2$ , $x_3$ , ... , $x_d$ )
\end{enumerate}



\section{Algorithms and Data Structures Review}

\subsection{Trees}

\begin{enumerate}
\item $2^l$
\item $2^{l+1}$ - 1
\end{enumerate}

\subsection{Common Runtimes}
\begin{enumerate}
\item O(nlgn)
\item O(n)
\item O(n)
\item O(nd)
\end{enumerate}

\subsection{Running times of code}
\begin{enumerate}
\item O(N)
\item O(N)
\item O(1)
\item O($N^2$)
\end{enumerate}



\end{document}
\documentclass{article}

\usepackage{fullpage}
\usepackage{color}
\usepackage{amsmath}
\usepackage{url}
\usepackage{verbatim}
\usepackage{graphicx}
\usepackage{parskip}
\usepackage{amssymb}
\usepackage{nicefrac}
\usepackage{listings} % For displaying code


\def\rubric#1{\gre{Rubric: \{#1\}}}{}

% Colors
\definecolor{blu}{rgb}{0,0,1}
\def\blu#1{{\color{blu}#1}}
\definecolor{gre}{rgb}{0,.5,0}
\def\gre#1{{\color{gre}#1}}
\definecolor{red}{rgb}{1,0,0}
\def\red#1{{\color{red}#1}}

% Math
\def\norm#1{\|#1\|}
\def\R{\mathbb{R}}
\def\argmax{\mathop{\rm arg\,max}}
\def\argmin{\mathop{\rm arg\,min}}
\newcommand{\mat}[1]{\begin{bmatrix}#1\end{bmatrix}}
\newcommand{\alignStar}[1]{\begin{align*}#1\end{align*}}

% LaTeX
\newcommand{\fig}[2]{\includegraphics[width=#1\textwidth]{a0f/#2}}
\newcommand{\centerfig}[2]{\begin{center}\includegraphics[width=#1\textwidth]{#2}\end{center}}
\def\items#1{\begin{itemize}#1\end{itemize}}
\def\enum#1{\begin{enumerate}#1\end{enumerate}}

\begin{document}

\title{CPSC 340 Assignment 0}
\author{Beverly Biwen Low - 58977463}
\date{}
\maketitle

\vspace{-4em}



\section{Linear Algebra Review}


\subsection{Basic Operations}

\enum{
\item 14
\item 0
\item (6 10 14)
\item square root of 5
\item (0 1 2)
\item $\left[\begin{array}{ccc}
3 & 1 & 1\\
2 & 3 & 1\\
2 & 1 & 3
\end{array}\right]$
\item $\left[\begin{array}{c}
6\\
5\\
7\\
\end{array}\right]$
}
\subsection{Matrix Algebra Rules}

\begin{enumerate}
\item True
\item True
\item False
\item False
\item False
\item True
\item False
\item True.
\item True
\end{enumerate}

\subsection{Special Matrices}
\enum{
\item Symmetric matrix is a square matrix that is equal to its transpose.
\item Identity matrix is a square matrix in which all elements of principal diagonal are ones and all other elements are zeros. Any matrix multiplied by it will get itself.
\item Orthogonal matrix is a matrix where the transpose is equal to the inverse.
}

\section{Probability Review}

\subsection{Rules of probability}

\begin{enumerate}
\item 0.25
\item 4 dollars
\item 0.55
\end{enumerate}

\subsection{Bayes Rule and Conditional Probability}

\begin{enumerate}
\item 0.010094
\item False positives
\item 0.00941153
\item Yes. The probability of a person is a drug user given that the test is true is very small.
\item 
\end{enumerate}


\section{Calculus Review}


\subsection{One-variable derivatives}

\begin{enumerate}
\item $\frac{14}{3}$
\item 0.25
\item 0
\item $\frac{-\exp(-x)}{1+\exp(-x)}$
\end{enumerate}

\subsection{Multi-variable derivatives}
\begin{enumerate}
\item ( $2x_1$ , $\exp(x_2)$ )
\item ( $\exp(x_1 + x_2x_3)$ , $x_3\exp(x_1 + x_2x_3)$ , $x_2\exp(x_1 + x_2x_3)$ )
\item ( $a_1$ , $a_2$ )
\item ( 4$x_1$ - 2$x_2$ , 2$x_2$ - 2$x_1$ )
\item ( $x_1$ , $x_2$ , $x_3$ , ... , $x_d$ )
\end{enumerate}



\section{Algorithms and Data Structures Review}

\subsection{Trees}

\begin{enumerate}
\item $2^l$
\item $2^{l+1}$ - 1
\end{enumerate}

\subsection{Common Runtimes}
\begin{enumerate}
\item O(nlgn)
\item O(n)
\item O(n)
\item O(nd)
\end{enumerate}

\subsection{Running times of code}
\begin{enumerate}
\item O(N)
\item O(N)
\item O(1)
\item O($N^2$)
\end{enumerate}



\end{document}
